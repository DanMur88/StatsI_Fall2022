\documentclass[12pt,letterpaper]{article}
\usepackage{graphicx,textcomp}
\usepackage{natbib}
\usepackage{setspace}
\usepackage{fullpage}
\usepackage{color}
\usepackage[reqno]{amsmath}
\usepackage{amsthm}
\usepackage{fancyvrb}
\usepackage{amssymb,enumerate}
\usepackage[all]{xy}
\usepackage{endnotes}
\usepackage{lscape}
\newtheorem{com}{Comment}
\usepackage{float}
\usepackage{hyperref}
\newtheorem{lem} {Lemma}
\newtheorem{prop}{Proposition}
\newtheorem{thm}{Theorem}
\newtheorem{defn}{Definition}
\newtheorem{cor}{Corollary}
\newtheorem{obs}{Observation}
\usepackage[compact]{titlesec}
\usepackage{dcolumn}
\usepackage{tikz}
\usetikzlibrary{arrows}
\usepackage{multirow}
\usepackage{xcolor}
\newcolumntype{.}{D{.}{.}{-1}}
\newcolumntype{d}[1]{D{.}{.}{#1}}
\definecolor{light-gray}{gray}{0.65}
\usepackage{url}
\usepackage{listings}
\usepackage{color}

\definecolor{codegreen}{rgb}{0,0.6,0}
\definecolor{codegray}{rgb}{0.5,0.5,0.5}
\definecolor{codepurple}{rgb}{0.58,0,0.82}
\definecolor{backcolour}{rgb}{0.95,0.95,0.92}

\lstdefinestyle{mystyle}{
	backgroundcolor=\color{backcolour},   
	commentstyle=\color{codegreen},
	keywordstyle=\color{magenta},
	numberstyle=\tiny\color{codegray},
	stringstyle=\color{codepurple},
	basicstyle=\footnotesize,
	breakatwhitespace=false,         
	breaklines=true,                 
	captionpos=b,                    
	keepspaces=true,                 
	numbers=left,                    
	numbersep=5pt,                  
	showspaces=false,                
	showstringspaces=false,
	showtabs=false,                  
	tabsize=2
}
\lstset{style=mystyle}
\newcommand{\Sref}[1]{Section~\ref{#1}}
\newtheorem{hyp}{Hypothesis}

\title{Problem Set 3}
\date{Due: November 20, 2022}
\author{Daniel Murray (13303981)}


\begin{document}
	\maketitle
	\section*{Instructions}
	\begin{itemize}
		\item Please show your work! You may lose points by simply writing in the answer. If the problem requires you to execute commands in \texttt{R}, please include the code you used to get your answers. Please also include the \texttt{.R} file that contains your code. If you are not sure if work needs to be shown for a particular problem, please ask.
	\item Your homework should be submitted electronically on GitHub.
	\item This problem set is due before 23:59 on Sunday November 20, 2022. No late assignments will be accepted.
	\item Total available points for this homework is 80.
	\end{itemize}

		\vspace{.25cm}
	
\noindent In this problem set, you will run several regressions and create an add variable plot (see the lecture slides) in \texttt{R} using the \texttt{incumbents\_subset.csv} dataset. Include all of your code.

	\vspace{.5cm}
\section*{Question 1}
\vspace{.25cm}
\noindent We are interested in knowing how the difference in campaign spending between incumbent and challenger affects the incumbent's vote share. 
	\begin{enumerate}
		\item Run a regression where the outcome variable is \texttt{voteshare} and the explanatory variable is \texttt{difflog}.\\
		
		\textbf{Answer:}\\
		
		A regression was run and summary statistics produced using the below code:
		
		\vspace{.5cm}
		
		\lstinputlisting[language=R, firstline=12, lastline=16]{PS03.R}  
		
		\vspace{.5cm}
		
		This produced the following output of summary statistics:
		
		\vspace{.5cm}
		
		% Table created by stargazer v.5.2.3 by Marek Hlavac, Social Policy Institute. E-mail: marek.hlavac at gmail.com
		% Date and time: Thu, Nov 10, 2022 - 16:05:59
		\begin{table}[!htbp] \centering 
			\caption{Summary Statistics of Regression Model} 
			\label{} 
			\begin{tabular}{@{\extracolsep{5pt}}lc} 
				\\[-1.8ex]\hline 
				\hline \\[-1.8ex] 
				& \multicolumn{1}{c}{\textit{Dependent variable:}} \\ 
				\cline{2-2} 
				\\[-1.8ex] & voteshare \\ 
				\hline \\[-1.8ex] 
				difflog & 0.042$^{***}$ \\ 
				& (0.001) \\ 
				& \\ 
				Constant & 0.579$^{***}$ \\ 
				& (0.002) \\ 
				& \\ 
				\hline \\[-1.8ex] 
				Observations & 3,193 \\ 
				R$^{2}$ & 0.367 \\ 
				Adjusted R$^{2}$ & 0.367 \\ 
				Residual Std. Error & 0.079 (df = 3191) \\ 
				F Statistic & 1,852.791$^{***}$ (df = 1; 3191) \\ 
				\hline 
				\hline \\[-1.8ex] 
				\textit{Note:}  & \multicolumn{1}{r}{$^{*}$p$<$0.1; $^{**}$p$<$0.05; $^{***}$p$<$0.01} \\ 
			\end{tabular} 
		\end{table} 
		
		\vspace{.5cm}
		
		\item Make a scatterplot of the two variables and add the regression line.\\
		
		\textbf{Answer:}\\
		
		A scatterplot was produced using the below code:
		
		\vspace{.5cm}
		
		\lstinputlisting[language=R, firstline=20, lastline=25]{PS03.R}  
		
		\vspace{.5cm}
		
		This produced the following scatterplot:
		
		\begin{figure}[H]\centering
			\caption{\footnotesize}
			\includegraphics[width=.75\textwidth]{Scatterplot voteshare ~ difflog.png}
		\end{figure} 
		
		\item Save the residuals of the model in a separate object.\\
		
		\textbf{Answer:}\\
		
		The residuals of the model were saved in a separate object using the below code:
		
		\vspace{.5cm}
		
		\lstinputlisting[language=R, firstline=29, lastline=30]{PS03.R}  
		
		\vspace{.5cm}
		
		\item Write the prediction equation.\\
		
		\textbf{Answer:}
		
		\[\hat{y} = \hat{\beta_0} + \hat{\beta_1}x_1 \]
		\[voteshare = 0.579 + 0.042\times difflog \]
		
	\end{enumerate}
	
\newpage

\section*{Question 2}
\noindent We are interested in knowing how the difference between incumbent and challenger's spending and the vote share of the presidential candidate of the incumbent's party are related.	\vspace{.25cm}
	\begin{enumerate}
		\item Run a regression where the outcome variable is \texttt{presvote} and the explanatory variable is \texttt{difflog}.\\
		
		\textbf{Answer:}\\
		
		A regression was run and summary statistics produced using the below code:
		
		\vspace{.5cm}
		
		\lstinputlisting[language=R, firstline=36, lastline=40]{PS03.R}  
		
		\vspace{.5cm}
		
		This produced the following output of summary statistics:
		
		\vspace{.5cm}
		
		% Table created by stargazer v.5.2.3 by Marek Hlavac, Social Policy Institute. E-mail: marek.hlavac at gmail.com
		% Date and time: Thu, Nov 10, 2022 - 16:38:57
		\begin{table}[!htbp] \centering 
			\caption{Summary Statistics of Regression Model} 
			\label{} 
			\begin{tabular}{@{\extracolsep{5pt}}lc} 
				\\[-1.8ex]\hline 
				\hline \\[-1.8ex] 
				& \multicolumn{1}{c}{\textit{Dependent variable:}} \\ 
				\cline{2-2} 
				\\[-1.8ex] & presvote \\ 
				\hline \\[-1.8ex] 
				difflog & 0.024$^{***}$ \\ 
				& (0.001) \\ 
				& \\ 
				Constant & 0.508$^{***}$ \\ 
				& (0.003) \\ 
				& \\ 
				\hline \\[-1.8ex] 
				Observations & 3,193 \\ 
				R$^{2}$ & 0.088 \\ 
				Adjusted R$^{2}$ & 0.088 \\ 
				Residual Std. Error & 0.110 (df = 3191) \\ 
				F Statistic & 307.715$^{***}$ (df = 1; 3191) \\ 
				\hline 
				\hline \\[-1.8ex] 
				\textit{Note:}  & \multicolumn{1}{r}{$^{*}$p$<$0.1; $^{**}$p$<$0.05; $^{***}$p$<$0.01} \\ 
			\end{tabular} 
		\end{table}
		
		\item Make a scatterplot of the two variables and add the regression line.\\
		
		\textbf{Answer:}\\
		
		A scatterplot was produced using the below code:
		
		\vspace{.5cm}
		
		\lstinputlisting[language=R, firstline=44, lastline=49]{PS03.R}  
		
		\vspace{.5cm}
		
		This produced the following scatterplot:
		
		\begin{figure}[H]\centering
			\caption{\footnotesize}
			\includegraphics[width=.75\textwidth]{Scatterplot presvote ~ difflog.png}
		\end{figure} 
		
		\item Save the residuals of the model in a separate object.\\
		
		\textbf{Answer:}\\
		
		The residuals of the model were saved in a separate object using the below code:
		
		\vspace{.5cm}
		
		\lstinputlisting[language=R, firstline=53, lastline=54]{PS03.R}  
		
		\vspace{.5cm}
		
		\item Write the prediction equation.\\
		
		\textbf{Answer:}
		
			\[\hat{y} = \hat{\beta_0} + \hat{\beta_1}x_1 \]
			\[presvote = 0.508 + 0.024\times difflog \]
		
	\end{enumerate}
	
	\newpage	
\section*{Question 3}

\noindent We are interested in knowing how the vote share of the presidential candidate of the incumbent's party is associated with the incumbent's electoral success.
	\vspace{.25cm}
	\begin{enumerate}
		\item Run a regression where the outcome variable is \texttt{voteshare} and the explanatory variable is \texttt{presvote}.\\
		
		\textbf{Answer:}\\
		
		A regression was run and summary statistics produced using the below code:
		
		\vspace{.5cm}
		
		\lstinputlisting[language=R, firstline=60, lastline=65]{PS03.R}  
		
		\vspace{.5cm}
		
		This produced the following output of summary statistics:
		
		\vspace{.5cm}
		
		% Table created by stargazer v.5.2.3 by Marek Hlavac, Social Policy Institute. E-mail: marek.hlavac at gmail.com
		% Date and time: Thu, Nov 10, 2022 - 21:22:51
		\begin{table}[!htbp] \centering 
			\caption{Summary Statistics of Regression Model} 
			\label{} 
			\begin{tabular}{@{\extracolsep{5pt}}lc} 
				\\[-1.8ex]\hline 
				\hline \\[-1.8ex] 
				& \multicolumn{1}{c}{\textit{Dependent variable:}} \\ 
				\cline{2-2} 
				\\[-1.8ex] & voteshare \\ 
				\hline \\[-1.8ex] 
				presvote & 0.388$^{***}$ \\ 
				& (0.013) \\ 
				& \\ 
				Constant & 0.441$^{***}$ \\ 
				& (0.008) \\ 
				& \\ 
				\hline \\[-1.8ex] 
				Observations & 3,193 \\ 
				R$^{2}$ & 0.206 \\ 
				Adjusted R$^{2}$ & 0.206 \\ 
				Residual Std. Error & 0.088 (df = 3191) \\ 
				F Statistic & 826.950$^{***}$ (df = 1; 3191) \\ 
				\hline 
				\hline \\[-1.8ex] 
				\textit{Note:}  & \multicolumn{1}{r}{$^{*}$p$<$0.1; $^{**}$p$<$0.05; $^{***}$p$<$0.01} \\ 
			\end{tabular} 
		\end{table} 
		
		\item Make a scatterplot of the two variables and add the regression line.\\
		
		\textbf{Answer:}\\
		
		A scatterplot was produced using the below code:
		
		\vspace{.5cm}
		
		\lstinputlisting[language=R, firstline=69, lastline=74]{PS03.R}  
		
		\vspace{.5cm}
		
		This produced the following scatterplot:
		
		\begin{figure}[H]\centering
			\caption{\footnotesize}
			\includegraphics[width=.75\textwidth]{Scatterplot voteshare ~ presvote.png}
		\end{figure} 
		
		\item Write the prediction equation.\\
		
		\textbf{Answer:}
		
		\[\hat{y} = \hat{\beta_0} + \hat{\beta_1}x_1 \]
		\[voteshare = 0.441 + 0.388\times presvote \]
		
	\end{enumerate}
	

\newpage	
\section*{Question 4}
\noindent The residuals from part (a) tell us how much of the variation in \texttt{voteshare} is $not$ explained by the difference in spending between incumbent and challenger. The residuals in part (b) tell us how much of the variation in \texttt{presvote} is $not$ explained by the difference in spending between incumbent and challenger in the district.
	\begin{enumerate}
		\item Run a regression where the outcome variable is the residuals from Question 1 and the explanatory variable is the residuals from Question 2.\\
		
		\textbf{Answer:}\\
		
		A regression was run and summary statistics produced using the below code:
		
		\vspace{.5cm}
		
		\lstinputlisting[language=R, firstline=80, lastline=85]{PS03.R}  
		
		\vspace{.5cm}
		
		This produced the following output of summary statistics:
		
		\vspace{.5cm}
		
		% Table created by stargazer v.5.2.3 by Marek Hlavac, Social Policy Institute. E-mail: marek.hlavac at gmail.com
		% Date and time: Thu, Nov 10, 2022 - 21:30:18
		\begin{table}[H] \centering 
			\caption{Summary Statistics of Regression Model} 
			\label{} 
			\begin{tabular}{@{\extracolsep{5pt}}lc} 
				\\[-1.8ex]\hline 
				\hline \\[-1.8ex] 
				& \multicolumn{1}{c}{\textit{Dependent variable:}} \\ 
				\cline{2-2} 
				\\[-1.8ex] & mod1\_res \\ 
				\hline \\[-1.8ex] 
				mod2\_res & 0.257$^{***}$ \\ 
				& (0.012) \\ 
				& \\ 
				Constant & $-$0.000 \\ 
				& (0.001) \\ 
				& \\ 
				\hline \\[-1.8ex] 
				Observations & 3,193 \\ 
				R$^{2}$ & 0.130 \\ 
				Adjusted R$^{2}$ & 0.130 \\ 
				Residual Std. Error & 0.073 (df = 3191) \\ 
				F Statistic & 476.975$^{***}$ (df = 1; 3191) \\ 
				\hline 
				\hline \\[-1.8ex] 
				\textit{Note:}  & \multicolumn{1}{r}{$^{*}$p$<$0.1; $^{**}$p$<$0.05; $^{***}$p$<$0.01} \\ 
			\end{tabular} 
		\end{table} 
		
		\item Make a scatterplot of the two residuals and add the regression line.\\
		
		\textbf{Answer:}\\
		
		A scatterplot was produced using the below code:
		
		\vspace{.5cm}
		
		\lstinputlisting[language=R, firstline=89, lastline=94]{PS03.R}  
		
		\vspace{.5cm}
		
		This produced the following scatterplot:
		
		\begin{figure}[H]\centering
			\caption{\footnotesize}
			\includegraphics[width=.75\textwidth]{Scatterplot mod1_res ~ mod2_res.png}
		\end{figure} 
		
		\item Write the prediction equation.\\
		
		
		\textbf{Answer:}
		
		\[\hat{y} = \hat{\beta_0} + \hat{\beta_1}x_1 \]
		\[mod1\_res = 0 + 0.257\times mod2\_res \]
		
	\end{enumerate}
	
	\newpage	

\section*{Question 5}
\noindent What if the incumbent's vote share is affected by both the president's popularity and the difference in spending between incumbent and challenger? 
	\begin{enumerate}
		\item Run a regression where the outcome variable is the incumbent's \texttt{voteshare} and the explanatory variables are \texttt{difflog} and \texttt{presvote}.\\
		
		\textbf{Answer:}\\
		
		A regression was run and summary statistics produced using the below code:
		
		\vspace{.5cm}
		
		\lstinputlisting[language=R, firstline=100, lastline=104]{PS03.R}  
		
		\vspace{.5cm}
		
		This produced the following output of summary statistics:
		
		\vspace{.4cm}
		
		% Table created by stargazer v.5.2.3 by Marek Hlavac, Social Policy Institute. E-mail: marek.hlavac at gmail.com
		% Date and time: Thu, Nov 10, 2022 - 21:56:00
		\begin{table}[H] \centering 
			\caption{Summary Statistics of Regression Model} 
			\label{} 
			\begin{tabular}{@{\extracolsep{5pt}}lc} 
				\\[-1.8ex]\hline 
				\hline \\[-1.8ex] 
				& \multicolumn{1}{c}{\textit{Dependent variable:}} \\ 
				\cline{2-2} 
				\\[-1.8ex] & voteshare \\ 
				\hline \\[-1.8ex] 
				difflog & 0.036$^{***}$ \\ 
				& (0.001) \\ 
				& \\ 
				presvote & 0.257$^{***}$ \\ 
				& (0.012) \\ 
				& \\ 
				Constant & 0.449$^{***}$ \\ 
				& (0.006) \\ 
				& \\ 
				\hline \\[-1.8ex] 
				Observations & 3,193 \\ 
				R$^{2}$ & 0.450 \\ 
				Adjusted R$^{2}$ & 0.449 \\ 
				Residual Std. Error & 0.073 (df = 3190) \\ 
				F Statistic & 1,302.947$^{***}$ (df = 2; 3190) \\ 
				\hline 
				\hline \\[-1.8ex] 
				\textit{Note:}  & \multicolumn{1}{r}{$^{*}$p$<$0.1; $^{**}$p$<$0.05; $^{***}$p$<$0.01} \\ 
			\end{tabular} 
		\end{table} 
		
		\item Write the prediction equation.\\
		
		\textbf{Answer:}
		
		\[\hat{y} = \hat{\beta_0} + \hat{\beta_1}x_1 + \hat{\beta_2}x_2 \]
		\[voteshare = 0.449 + 0.036\times difflog + 0.257\times presvote\]
		
		\vspace{.5cm}
		
		\item What is it in this output that is identical to the output in Question 4? Why do you think this is the case?\\
		
		\textbf{Answer:}\\
		
		The coefficient for \texttt{presvote} in this question has the same value as the coefficient for \texttt{mod2\_res} in Question 4. The residuals from Question 1 represent the variation in \texttt{voteshare} that is not explained by \texttt{difflog}, and the residuals from Question 2 represent the variation in \texttt{presvote} that is not explained by \texttt{difflog}. Therefore, the coefficient for \texttt{mod2\_res} in Question 4 represents the amount of co-variation between \texttt{presvote} and \texttt{voteshare} that is not explained by \texttt{difflog}.
		
	\end{enumerate}




\end{document}
